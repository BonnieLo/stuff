\documentclass[12pt,reqno]{article}

\usepackage[usenames]{color}
\usepackage{amssymb}
\usepackage{graphicx}
\usepackage{amscd}
\usepackage{wrapfig}

\usepackage[colorlinks=true,
linkcolor=black, filecolor=black,
citecolor=black]{hyperref}

\usepackage{fullpage}
\usepackage{float}

%\usepackage{psfig}
\usepackage{graphics,amsmath,amssymb,amsthm}
\usepackage{amsfonts}
\usepackage{latexsym}
\usepackage{epsf}

\DeclareGraphicsExtensions{.png}
\renewcommand{\figurename}{Fig.}

\begin{document}

\begin{center}
 \vskip 1cm{\LARGE\bf State machines for Python}
 \vskip 1cm
 \large

Will Ware \\
12 Francine Road \\
Framingham, MA \\
\href{mailto:wware@alum.mit.edu}{\tt wware@alum.mit.edu}
\end{center}


\vskip .2 in

The behavior of some systems can be modeled as sets of state machines.
Suitably implemented, the same state machines can be used for an
application running in real time and for simulation and verification.

This library is such an implementation in Python. State machines are
space efficient and it is practical to maintain thousands. This has
potential application in computer games and in animations involving
flocking species such as birds or fish.

State machines as presented here have no intrinsic concept of time but
can post events in the future, and listen for those events later.
Timely delivery of future events is managed by a framework in which
state machines run, called a Runner. A state machine can wait for
input. When doing so, it will remain at the same state until the new
input occurs.

There are two Runner implementations, one for real time and one for
simulation. When in simulation, the Runner will compress time
intervals where nothing is happening so that simulations run more
quickly.

The Runner manages a group of state machines as well as a queue of
timestamped future events. When a state machine needs to specify a
time interval, it does so by stamping an event with the future time,
and using the event to change its own input. ``Events'' are simply
Python functions with no arguments.

Using GraphViz it is easy to make nice diagrams of state machines. The
three examples below (Figures \ref{ex1}, \ref{ex2}, and \ref{ex3}) are
taken from the smtests.py source file.

\begin{figure}
  \begin{center}
  \includegraphics[scale=0.65]{ex1.eps}
  \caption{First example state machine}
  \label{ex1}
  \end{center}
\end{figure}

\begin{figure}
  \begin{center}
  \includegraphics[scale=0.65]{ex2.eps}
  \caption{Second example state machine}
  \label{ex2}
  \end{center}
\end{figure}

\begin{figure}
  \begin{center}
  \includegraphics[scale=0.65]{ex3.eps}
  \caption{Third example state machine}
  \label{ex3}
  \end{center}
\end{figure}

\end{document}
